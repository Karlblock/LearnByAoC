\chapter{Reference Python}
\label{chap:python_ref}

% ============================================================
\section{Collections}
% ============================================================

\begin{lstlisting}[caption={Imports essentiels}]
from collections import deque, Counter, defaultdict
from itertools import permutations, combinations, product
from functools import lru_cache, reduce
import heapq
import math
import re
\end{lstlisting}

% ============================================================
\section{Comprehensions}
% ============================================================

\begin{lstlisting}[caption={Syntaxe des comprehensions}]
# Liste
[x*2 for x in range(10) if x % 2 == 0]

# Dict
{k: v for k, v in items}

# Set
{x % 3 for x in range(10)}

# Generator
(x*2 for x in range(10))
\end{lstlisting}

% ============================================================
\section{Fonctions utiles}
% ============================================================

\begin{tabular}{ll}
\toprule
\textbf{Fonction} & \textbf{Usage} \\
\midrule
\texttt{enumerate()} & Indices + valeurs \\
\texttt{zip()} & Parcourir plusieurs listes \\
\texttt{map()} & Appliquer une fonction \\
\texttt{filter()} & Filtrer des elements \\
\texttt{sorted()} & Trier (renvoie nouvelle liste) \\
\texttt{reversed()} & Inverser \\
\texttt{any()} / \texttt{all()} & Tests logiques \\
\bottomrule
\end{tabular}
