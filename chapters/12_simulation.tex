\chapter{Simulation : Modeliser le Monde}
\label{chap:simulation}

\begin{histoire}
Le Jeu de la Vie de Conway : des cellules naissent et meurent selon des regles simples. Pourtant, des structures complexes emergent. Bienvenue dans la simulation !
\end{histoire}

% ============================================================
\section{Automates cellulaires}
% ============================================================

\begin{lstlisting}[caption={Game of Life simplifie}]
def step(grille):
    nouvelle = set()
    
    # Compter les voisins
    for cell in grille:
        voisins = count_neighbors(cell, grille)
        if voisins in [2, 3]:
            nouvelle.add(cell)
    
    # Naissance
    for cell in candidates(grille):
        if count_neighbors(cell, grille) == 3:
            nouvelle.add(cell)
    
    return nouvelle
\end{lstlisting}

% ============================================================
\section{Detection de cycles}
% ============================================================

\begin{lstlisting}[caption={Trouver un cycle}]
def find_cycle(initial_state, step_func):
    seen = {initial_state: 0}
    state = initial_state
    step = 0
    
    while True:
        state = step_func(state)
        step += 1
        
        if state in seen:
            cycle_start = seen[state]
            cycle_length = step - cycle_start
            return cycle_start, cycle_length
        
        seen[state] = step
\end{lstlisting}

% ============================================================
\section{Challenges d'application}
% ============================================================

\begin{exercice}
\textbf{AoC 2020 Day 17 - Conway Cubes}

Game of Life en 3D et 4D.

\textbf{Concepts :} Simulation, dimensions multiples
\end{exercice}

\begin{cyber}
\begin{itemize}
    \item \textbf{Analyse de malware} : Execution symbolique
    \item \textbf{Fuzzing} : Simulation d'entrees
\end{itemize}
\end{cyber}
