\chapter{Dictionnaires : Associer Cles et Valeurs}
\label{chap:dictionnaires}

\begin{histoire}
Un dictionnaire, c'est comme ton carnet de contacts : tu cherches un nom (la cle) et tu trouves le numero (la valeur). Instantanement.
\end{histoire}

% ============================================================
\section{Qu'est-ce qu'un dict ?}
% ============================================================

\begin{definition}[Dictionnaire]
Un \texttt{dict} associe des \textbf{cles} a des \textbf{valeurs}. Acces en O(1).
\end{definition}

\begin{lstlisting}[caption={Operations de base}]
# Creation
scores = {"Alice": 100, "Bob": 85}

# Acces
scores["Alice"]      # 100
scores.get("Eve", 0) # 0 (defaut si absent)

# Modification
scores["Alice"] = 110
scores["Eve"] = 95   # Nouvelle entree

# Iteration
for nom, score in scores.items():
    print(f"{nom}: {score}")
\end{lstlisting}

% ============================================================
\section{defaultdict : le dict intelligent}
% ============================================================

\begin{lstlisting}[caption={defaultdict pour eviter les KeyError}]
from collections import defaultdict

# Comptage
compteur = defaultdict(int)
for mot in mots:
    compteur[mot] += 1  # Pas besoin d'initialiser !

# Listes de valeurs
graphe = defaultdict(list)
graphe["A"].append("B")  # Pas besoin de verifier si "A" existe
\end{lstlisting}

% ============================================================
\section{Counter : compter automatiquement}
% ============================================================

\begin{lstlisting}[caption={Counter pour les frequences}]
from collections import Counter

texte = "abracadabra"
freq = Counter(texte)
# Counter({'a': 5, 'b': 2, 'r': 2, 'c': 1, 'd': 1})

freq.most_common(3)  # [('a', 5), ('b', 2), ('r', 2)]
\end{lstlisting}

% ============================================================
\section{Challenges d'application}
% ============================================================

\begin{exercice}
\textbf{AoC 2024 Day 1 - Historian Hysteria}

Compte les occurrences et calcule un score de similarite.

\textbf{Concepts :} Counter, multiplication
\end{exercice}

\begin{cyber}
\begin{itemize}
    \item \textbf{Analyse de frequence} : Casser le chiffrement par substitution
    \item \textbf{Cache/Memoization} : Accelerer les calculs
\end{itemize}
\end{cyber}
