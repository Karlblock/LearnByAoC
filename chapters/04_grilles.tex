\chapter{Grilles 2D : Naviguer dans l'Espace}
\label{chap:grilles}

\begin{histoire}
Tu connais les echecs ? Une grille 8x8 ou chaque piece peut bouger dans certaines directions. En programmation, on utilise constamment des grilles !
\end{histoire}

% ============================================================
\section{Representer une grille}
% ============================================================

\begin{lstlisting}[caption={Creer et parser une grille}]
# Parser un fichier en grille
with open("input.txt") as f:
    grille = [list(ligne.strip()) for ligne in f]

# Acces : grille[ligne][colonne]
grille[0][0]  # coin haut-gauche

# Dimensions
hauteur = len(grille)
largeur = len(grille[0])
\end{lstlisting}

\begin{piege}
Attention : \texttt{grille[ligne][colonne]} = \texttt{grille[y][x]} !
\end{piege}

% ============================================================
\section{Les 4 directions}
% ============================================================

\begin{lstlisting}[caption={Directions cardinales}]
DIRECTIONS = [
    (-1, 0),  # Haut
    (1, 0),   # Bas
    (0, -1),  # Gauche
    (0, 1)    # Droite
]
\end{lstlisting}

% ============================================================
\section{Distance de Manhattan}
% ============================================================

\begin{definition}[Distance de Manhattan]
Distance en blocs : $d = |x_1 - x_2| + |y_1 - y_2|$
\end{definition}

\begin{lstlisting}[caption={Calcul Manhattan}]
def manhattan(p1, p2):
    return abs(p1[0] - p2[0]) + abs(p1[1] - p2[1])
\end{lstlisting}

% ============================================================
\section{Challenges d'application}
% ============================================================

\begin{exercice}
\textbf{AoC 2015 Day 3 - Perfectly Spherical Houses}

Suis des directions et compte les maisons visitees.

\textbf{Concepts :} Directions, set de positions
\end{exercice}

\begin{cyber}
\begin{itemize}
    \item \textbf{Analyse d'images} : Steganographie, detection
    \item \textbf{Cartographie reseau} : Visualisation de topologie
\end{itemize}
\end{cyber}
