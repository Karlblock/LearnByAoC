\chapter{Complexite Algorithmique}
\label{chap:complexite}

% ============================================================
\section{Notation Big-O}
% ============================================================

\begin{definition}[Big-O]
La complexite decrit comment le temps d'execution evolue avec la taille de l'entree.
\end{definition}

% ============================================================
\section{Complexites courantes}
% ============================================================

\begin{tabular}{llp{6cm}}
\toprule
\textbf{Complexite} & \textbf{Nom} & \textbf{Exemples} \\
\midrule
O(1) & Constante & Acces dict/set, operations mathematiques \\
O(log n) & Logarithmique & Recherche binaire, operations sur heap \\
O(n) & Lineaire & Parcours de liste, recherche lineaire \\
O(n log n) & Linearithmique & Tri (merge sort, timsort) \\
O(n\textsuperscript{2}) & Quadratique & Double boucle imbriquee \\
O(n\textsuperscript{3}) & Cubique & Triple boucle, multiplication de matrices \\
O(2\textsuperscript{n}) & Exponentielle & Bruteforce, backtracking naif \\
O(n!) & Factorielle & Permutations completes \\
\bottomrule
\end{tabular}

% ============================================================
\section{Regles de calcul}
% ============================================================

\begin{enumerate}
    \item On garde le terme dominant : O(n\textsuperscript{2} + n) = O(n\textsuperscript{2})
    \item On ignore les constantes : O(3n) = O(n)
    \item Boucles imbriquees : on multiplie
    \item Boucles sequentielles : on additionne
\end{enumerate}

% ============================================================
\section{Complexite spatiale}
% ============================================================

\begin{definition}[Complexite spatiale]
Quantite de memoire utilisee en fonction de la taille de l'entree.
\end{definition}

\begin{itemize}
    \item Liste de n elements : O(n)
    \item Grille n x n : O(n\textsuperscript{2})
    \item Set de positions visitees : O(nombre de positions uniques)
\end{itemize}
