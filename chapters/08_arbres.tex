\chapter{Arbres : Structures Hierarchiques}
\label{chap:arbres}

\begin{histoire}
Un arbre genealogique : grand-parents en haut, parents au milieu, enfants en bas. Chaque personne a un parent (sauf la racine) et peut avoir plusieurs enfants.
\end{histoire}

% ============================================================
\section{Qu'est-ce qu'un arbre ?}
% ============================================================

\begin{definition}[Arbre]
Structure hierarchique avec une \textbf{racine}, des \textbf{noeuds} et des \textbf{feuilles} (noeuds sans enfants).
\end{definition}

\begin{lstlisting}[caption={Representation avec dict}]
# Arbre comme dict parent -> enfants
arbre = {
    "COM": ["B"],
    "B": ["C", "G"],
    "C": ["D"],
    "G": ["H"],
}

# Ou enfant -> parent (pour remonter)
parents = {
    "B": "COM",
    "C": "B",
    "D": "C",
}
\end{lstlisting}

% ============================================================
\section{Parcours d'arbre}
% ============================================================

\begin{lstlisting}[caption={Parcours recursif}]
def parcours(noeud, arbre, profondeur=0):
    print("  " * profondeur + noeud)
    for enfant in arbre.get(noeud, []):
        parcours(enfant, arbre, profondeur + 1)
\end{lstlisting}

% ============================================================
\section{Challenges d'application}
% ============================================================

\begin{exercice}
\textbf{AoC 2019 Day 6 - Universal Orbit Map}

Compte les orbites directes et indirectes dans un arbre.

\textbf{Concepts :} Parcours d'arbre, comptage de profondeur
\end{exercice}

\begin{cyber}
\begin{itemize}
    \item \textbf{Systeme de fichiers} : Arborescence de repertoires
    \item \textbf{DOM/XML} : Parsing de documents structures
\end{itemize}
\end{cyber}
