\chapter{Index des Challenges}
\label{chap:index_challenges}

% ============================================================
\section{Par Concept Principal}
% ============================================================

\subsection{Parsing et I/O}
\begin{itemize}
    \item 2015 Day 1-5 : Parsing basique
    \item 2020 Day 4 : Parsing complexe (passeports)
    \item 2016 Day 4 : Regex
\end{itemize}

\subsection{Grilles et Navigation}
\begin{itemize}
    \item 2015 Day 3 : Set de positions
    \item 2015 Day 6 : Grille booleenne
    \item 2020 Day 11 : Game of Life
    \item 2016 Day 1 : Distance Manhattan
\end{itemize}

\subsection{BFS / Pathfinding}
\begin{itemize}
    \item 2016 Day 13 : BFS dans labyrinthe
    \item 2022 Day 12 : BFS avec contraintes
    \item 2021 Day 15 : Dijkstra
    \item 2018 Day 22 : A* avance
\end{itemize}

\subsection{Recursion et DP}
\begin{itemize}
    \item 2015 Day 7 : Recursion avec memoization
    \item 2020 Day 10 : DP comptage
    \item 2015 Day 17 : Backtracking
\end{itemize}

\subsection{Mathematiques}
\begin{itemize}
    \item 2020 Day 13 : Theoreme chinois des restes
    \item 2019 Day 12 : LCM pour cycles
    \item 2015 Day 4 : Hachage MD5
\end{itemize}

% ============================================================
\section{Par Difficulte}
% ============================================================

\subsection{Niveau Debutant (\faStar)}
2015 Days 1-5, 2020 Day 1, 2024 Day 1

\subsection{Niveau Intermediaire (\faStar\faStar)}
2016 Day 13, 2015 Day 6-10, 2020 Days 3-8

\subsection{Niveau Avance (\faStar\faStar\faStar)}
2021 Day 15, 2015 Day 7, 2020 Days 10-15

\subsection{Niveau Expert (\faStar\faStar\faStar\faStar)}
2020 Day 13 (CRT), Days 20-25 de toutes les annees

% ============================================================
\section{Challenges Emblematiques}
% ============================================================

\begin{tabular}{lp{8cm}}
\toprule
\textbf{Challenge} & \textbf{Pourquoi c'est un classique} \\
\midrule
2015 Day 7 & Introduction parfaite a la recursion avec memoization \\
2016 Day 13 & BFS pur dans un espace genere \\
2021 Day 15 & Meilleur challenge pour apprendre Dijkstra \\
2020 Day 13 & Le theoreme chinois des restes en pratique \\
2020 Day 17 & Game of Life en N dimensions \\
\bottomrule
\end{tabular}
