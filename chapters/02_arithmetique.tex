\chapter{Arithmetique : Compter et Calculer}
\label{chap:arithmetique}

\begin{histoire}
Tu te souviens quand tu as appris a compter sur tes doigts ? En programmation, on fait pareil - mais avec des millions de nombres, et beaucoup plus vite !
\end{histoire}

% ============================================================
\section{Les operations de base}
% ============================================================

\begin{definition}[Operateurs arithmetiques]
Python utilise les operateurs classiques :
\begin{itemize}
    \item \texttt{+} : Addition
    \item \texttt{-} : Soustraction
    \item \texttt{*} : Multiplication
    \item \texttt{/} : Division (resultat flottant)
    \item \texttt{//} : Division entiere
    \item \texttt{\%} : Modulo (reste de la division)
    \item \texttt{**} : Puissance
\end{itemize}
\end{definition}

\begin{lstlisting}[caption={Operations arithmetiques}]
17 / 5    # 3.4   (division flottante)
17 // 5   # 3     (division entiere)
17 % 5    # 2     (reste : 17 = 3*5 + 2)
2 ** 10   # 1024  (2 puissance 10)
\end{lstlisting}

% ============================================================
\section{Le modulo : ton nouvel ami}
% ============================================================

\begin{concept}
Le \textbf{modulo} (\texttt{\%}) donne le \textbf{reste} de la division. C'est l'operation la plus utile en programmation !
\end{concept}

\begin{lstlisting}[caption={Usages courants du modulo}]
# Pair ou impair ?
n % 2 == 0  # True si n est pair

# Position dans un cycle (0 a n-1)
jour = total_jours % 7  # Jour de la semaine (0-6)

# Eviter les depassements de liste
index = i % len(liste)  # Revient au debut si depasse

# Extraire le dernier chiffre
dernier = n % 10  # Chiffre des unites
\end{lstlisting}

\begin{exercice}
\textbf{AoC 2015 Day 1 - Not Quite Lisp}

Utilise un compteur simple : \texttt{(} ajoute 1, \texttt{)} soustrait 1. Trouve l'etage final.

\textbf{Concepts :} Iteration, comptage, conditions
\end{exercice}

% ============================================================
\section{GCD et LCM : cycles et periodicite}
% ============================================================

\begin{definition}[PGCD et PPCM]
\begin{itemize}
    \item \textbf{PGCD} (GCD) : Plus Grand Commun Diviseur
    \item \textbf{PPCM} (LCM) : Plus Petit Commun Multiple
\end{itemize}
\end{definition}

\begin{lstlisting}[caption={Calcul du LCM}]
import math

def lcm(a, b):
    return a * b // math.gcd(a, b)

# Quand 3 cycles de periodes 4, 6, 9 se synchronisent ?
from functools import reduce
periodes = [4, 6, 9]
sync = reduce(lcm, periodes)  # 36
\end{lstlisting}

\begin{cyber}
Le PGCD est utilise en \textbf{cryptographie} (RSA) et le PPCM pour analyser les \textbf{cycles} dans les malwares.
\end{cyber}

% ============================================================
\section{Challenges d'application}
% ============================================================

\begin{exercice}
\textbf{AoC 2020 Day 13 - Shuttle Search}

Trouve quand plusieurs bus avec des periodes differentes passent en meme temps.

\textbf{Difficulte :} \faStar\faStar\faStar\faStar

\textbf{Concepts :} LCM, modulo, arithmetique modulaire
\end{exercice}

% ============================================================
\section{Resume}
% ============================================================

\begin{tabular}{ll}
\toprule
\textbf{Concept} & \textbf{Usage principal} \\
\midrule
\texttt{\%} (modulo) & Cycles, parite, position circulaire \\
\texttt{//} (div. entiere) & Quotient sans decimales \\
\texttt{math.gcd()} & Plus grand diviseur commun \\
\texttt{math.lcm()} & Synchronisation de cycles \\
\bottomrule
\end{tabular}
