\chapter{Recherche : Trouver ce qu'on Cherche}
\label{chap:recherche}

\begin{histoire}
Imagine que tu cherches un mot dans le dictionnaire. Tu ne lis pas toutes les pages depuis le debut ! Tu ouvres au milieu, tu regardes si c'est avant ou apres, et tu recommences. C'est la \textbf{recherche binaire}.
\end{histoire}

% ============================================================
\section{Recherche lineaire}
% ============================================================

\begin{definition}[Recherche lineaire]
Parcourir tous les elements un par un jusqu'a trouver ce qu'on cherche. Complexite : O(n).
\end{definition}

% ============================================================
\section{Recherche dans un set : O(1)}
% ============================================================

\begin{concept}
Un \texttt{set} utilise une \textbf{table de hachage}. La recherche est instantanee (O(1)).
\end{concept}

\begin{lstlisting}[caption={Comparaison list vs set}]
# LENT : recherche dans une liste O(n)
if x in liste:  # parcourt toute la liste
    ...

# RAPIDE : recherche dans un set O(1)
ensemble = set(liste)
if x in ensemble:  # acces direct
    ...
\end{lstlisting}

% ============================================================
\section{Recherche binaire}
% ============================================================

\begin{definition}[Recherche binaire]
Diviser l'espace de recherche en deux a chaque etape. Complexite : O(log n).
\end{definition}

\begin{lstlisting}[caption={Module bisect}]
import bisect

liste = [1, 3, 5, 7, 9]
bisect.bisect_left(liste, 6)   # 3 (position d'insertion)
\end{lstlisting}

% ============================================================
\section{Challenges d'application}
% ============================================================

\begin{exercice}
\textbf{AoC 2020 Day 1 - Report Repair}

Trouve deux nombres dont la somme est 2020. Optimise avec un set.

\textbf{Concepts :} Recherche dans set, complement
\end{exercice}

\begin{cyber}
\begin{itemize}
    \item \textbf{Bruteforce intelligent} : Recherche binaire sur l'espace de cles
    \item \textbf{Analyse de logs} : Recherche rapide dans des GB de logs
\end{itemize}
\end{cyber}

% ============================================================
\section{Resume}
% ============================================================

\begin{tabular}{lll}
\toprule
\textbf{Methode} & \textbf{Complexite} & \textbf{Quand} \\
\midrule
Liste + \texttt{in} & O(n) & Petite liste \\
Set + \texttt{in} & O(1) & Recherches multiples \\
Recherche binaire & O(log n) & Donnees triees \\
\bottomrule
\end{tabular}
