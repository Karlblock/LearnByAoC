\chapter{Piles et Files : L'Ordre de Traitement}
\label{chap:piles_files}

\begin{histoire}
Une pile d'assiettes : tu poses en haut, tu prends en haut (LIFO). Une file d'attente au cinema : le premier arrive est le premier servi (FIFO).
\end{histoire}

% ============================================================
\section{La Pile (Stack) : LIFO}
% ============================================================

\begin{definition}[Pile]
Last In, First Out. Le dernier element ajoute est le premier retire.
\end{definition}

\begin{lstlisting}[caption={Pile avec une liste}]
pile = []
pile.append(1)  # Push
pile.append(2)
pile.pop()      # 2 (Pop)
pile[-1]        # 1 (Peek sans retirer)
\end{lstlisting}

% ============================================================
\section{La File (Queue) : FIFO}
% ============================================================

\begin{definition}[File]
First In, First Out. Le premier element ajoute est le premier retire.
\end{definition}

\begin{lstlisting}[caption={File avec deque}]
from collections import deque

file = deque()
file.append(1)     # Ajouter a droite
file.append(2)
file.popleft()     # 1 (Retirer a gauche)
\end{lstlisting}

\begin{concept}
\texttt{deque} est O(1) aux deux extremites. Une liste est O(n) pour \texttt{pop(0)} !
\end{concept}

% ============================================================
\section{Applications}
% ============================================================

\begin{itemize}
    \item \textbf{Pile} : Parentheses equilibrees, DFS, undo/redo
    \item \textbf{File} : BFS, traitement dans l'ordre d'arrivee
\end{itemize}

% ============================================================
\section{Challenges d'application}
% ============================================================

\begin{exercice}
\textbf{AoC 2021 Day 10 - Syntax Scoring}

Verifie l'equilibrage des parentheses avec une pile.

\textbf{Concepts :} Pile, matching de caracteres
\end{exercice}

\begin{cyber}
\begin{itemize}
    \item \textbf{Buffer overflow} : Comprendre la pile d'execution
    \item \textbf{Message queues} : Communication inter-processus
\end{itemize}
\end{cyber}
