\chapter{Optimisation : Faire Plus Vite}
\label{chap:optimisation}

\begin{histoire}
Ton programme fonctionne, mais il met 10 minutes. Comment le faire tourner en 10 secondes ? L'optimisation, c'est l'art de faire plus avec moins.
\end{histoire}

% ============================================================
\section{Identifier les goulots}
% ============================================================

\begin{lstlisting}[caption={Profiling simple}]
import time

start = time.time()
# Code a mesurer
result = slow_function()
print(f"Temps: {time.time() - start:.2f}s")
\end{lstlisting}

% ============================================================
\section{Complexite algorithmique}
% ============================================================

\begin{tabular}{lll}
\toprule
\textbf{Complexite} & \textbf{Exemple} & \textbf{1M elements} \\
\midrule
O(1) & Acces dict & Instantane \\
O(log n) & Recherche binaire & 20 ops \\
O(n) & Parcours liste & 1M ops \\
O(n log n) & Tri & 20M ops \\
O(n\textsuperscript{2}) & Double boucle & 1T ops \\
O(2\textsuperscript{n}) & Bruteforce & Impossible \\
\bottomrule
\end{tabular}

% ============================================================
\section{Techniques courantes}
% ============================================================

\begin{itemize}
    \item Utiliser \texttt{set} au lieu de \texttt{list} pour les recherches
    \item Memoization pour eviter les recalculs
    \item Eviter les copies inutiles de listes
    \item Utiliser des generateurs au lieu de listes
\end{itemize}

% ============================================================
\section{Challenges d'application}
% ============================================================

\begin{exercice}
\textbf{AoC 2025 Day 7 - Laboratories}

Optimise une simulation exponentielle avec \texttt{defaultdict(int)}.

\textbf{Concepts :} Comptage au lieu d'enumeration
\end{exercice}

\begin{cyber}
\begin{itemize}
    \item \textbf{Timing attacks} : Mesurer les differences de temps
    \item \textbf{DoS} : Exploiter la complexite algorithmique
\end{itemize}
\end{cyber}
