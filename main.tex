\documentclass[11pt,a4paper,twoside]{book}

% === PACKAGES ===
\usepackage[utf8]{inputenc}
\usepackage[T1]{fontenc}
\usepackage[french]{babel}
\usepackage{amsmath,amssymb,amsthm}
\usepackage{algorithm}
\usepackage{algpseudocode}
\usepackage{listings}
\usepackage{xcolor}
\usepackage{graphicx}
\usepackage{hyperref}
\usepackage{tikz}
\usetikzlibrary{arrows.meta, positioning, shapes.geometric}
\usepackage{tcolorbox}
\usepackage{fontawesome5}
\usepackage{enumitem}
\usepackage{booktabs}
\usepackage{multirow}
\usepackage{fancyhdr}
\usepackage{geometry}
\usepackage{qrcode}        % QR codes pour liens interactifs

% === GEOMETRY ===
\geometry{
    left=2.5cm,
    right=2.5cm,
    top=2.5cm,
    bottom=2.5cm
}

% === COLORS ===
\definecolor{codegreen}{rgb}{0,0.6,0}
\definecolor{codegray}{rgb}{0.5,0.5,0.5}
\definecolor{codepurple}{rgb}{0.58,0,0.82}
\definecolor{backcolour}{rgb}{0.95,0.95,0.92}
\definecolor{cyberblue}{RGB}{0,120,215}
\definecolor{cyberorange}{RGB}{255,140,0}
\definecolor{cybergreen}{RGB}{0,180,80}

% === CODE LISTINGS ===
\lstdefinestyle{python}{
    backgroundcolor=\color{backcolour},
    commentstyle=\color{codegreen},
    keywordstyle=\color{cyberblue}\bfseries,
    numberstyle=\tiny\color{codegray},
    stringstyle=\color{codepurple},
    basicstyle=\ttfamily\footnotesize,
    breakatwhitespace=false,
    breaklines=true,
    captionpos=b,
    keepspaces=true,
    numbers=left,
    numbersep=5pt,
    showspaces=false,
    showstringspaces=false,
    showtabs=false,
    tabsize=4,
    language=Python
}
\lstset{style=python}

% === PSEUDOCODE ===
\algrenewcommand\algorithmicrequire{\textbf{Entree:}}
\algrenewcommand\algorithmicensure{\textbf{Sortie:}}
\algrenewcommand\algorithmicif{\textbf{Si}}
\algrenewcommand\algorithmicthen{\textbf{alors}}
\algrenewcommand\algorithmicelse{\textbf{sinon}}
\algrenewcommand\algorithmicfor{\textbf{Pour}}
\algrenewcommand\algorithmicwhile{\textbf{Tant que}}
\algrenewcommand\algorithmicdo{\textbf{faire}}
\algrenewcommand\algorithmicreturn{\textbf{Retourner}}

% === CUSTOM BOXES ===
\tcbuselibrary{skins,breakable}

% Box: Histoire (pour enfants)
\newtcolorbox{histoire}{
    colback=cyan!5,
    colframe=cyberblue,
    fonttitle=\bfseries,
    title={\faBookOpen\ L'Histoire},
    breakable
}

% Box: Concept cle
\newtcolorbox{concept}{
    colback=green!5,
    colframe=cybergreen,
    fonttitle=\bfseries,
    title={\faLightbulb\ Concept Cle},
    breakable
}

% Box: Piege a eviter
\newtcolorbox{piege}{
    colback=red!5,
    colframe=red!70!black,
    fonttitle=\bfseries,
    title={\faExclamationTriangle\ Piege a Eviter},
    breakable
}

% Box: Application cyber
\newtcolorbox{cyber}{
    colback=orange!5,
    colframe=cyberorange,
    fonttitle=\bfseries,
    title={\faShield*\ En Cybersecurite},
    breakable
}

% Box: Exercice
\newtcolorbox{exercice}{
    colback=gray!5,
    colframe=gray!70!black,
    fonttitle=\bfseries,
    title={\faPencil*\ Exercice},
    breakable
}

% === COMMANDES INTERACTIVES ===
% Lien Replit avec QR code
\newcommand{\replit}[2]{%
    \begin{tcolorbox}[
        colback=blue!5,
        colframe=cyberblue,
        fonttitle=\bfseries,
        title={\faCode\ Tester en ligne},
        boxrule=1pt
    ]
    \begin{minipage}{0.75\textwidth}
        \textbf{#1}\\[0.3em]
        \url{#2}\\[0.5em]
        \small Scanne le QR code ou clique sur le lien pour ouvrir l'environnement de code.
    \end{minipage}%
    \hfill
    \begin{minipage}{0.2\textwidth}
        \centering
        \qrcode[height=2cm]{#2}
    \end{minipage}
    \end{tcolorbox}
}

% Lien Colab (pour notebooks)
\newcommand{\colab}[2]{%
    \begin{tcolorbox}[
        colback=yellow!5,
        colframe=orange!80!black,
        fonttitle=\bfseries,
        title={\faGoogle\ Google Colab},
        boxrule=1pt
    ]
    \begin{minipage}{0.75\textwidth}
        \textbf{#1}\\[0.3em]
        \url{#2}
    \end{minipage}%
    \hfill
    \begin{minipage}{0.2\textwidth}
        \centering
        \qrcode[height=2cm]{#2}
    \end{minipage}
    \end{tcolorbox}
}

% Version inline (juste le QR + lien)
\newcommand{\codelink}[1]{%
    \marginpar{%
        \centering
        \qrcode[height=1.5cm]{#1}\\
        \tiny\url{#1}
    }%
}

% === THEOREMS ===
\theoremstyle{definition}
\newtheorem{definition}{Definition}[chapter]
\newtheorem{exemple}{Exemple}[chapter]
\newtheorem{propriete}{Propriete}[chapter]

% === HEADERS ===
\pagestyle{fancy}
\fancyhf{}
\fancyhead[LE,RO]{\thepage}
\fancyhead[LO]{\nouppercase{\rightmark}}
\fancyhead[RE]{\nouppercase{\leftmark}}

% === META ===
\title{
    \Huge\textbf{Apprendre la Cyber par les Challenges}\\[0.5cm]
    \Large Du Lycee a l'Universite\\[1cm]
    \IfFileExists{assets/logo.png}{%
        \includegraphics[width=0.4\textwidth]{assets/logo.png}
    }{%
        \begin{tikzpicture}
            \node[draw=cyberblue, line width=2pt, rounded corners=10pt,
                  inner sep=15pt, font=\Huge\bfseries\color{cyberblue}] {LearnByAoC};
        \end{tikzpicture}
    }
}
\author{
    \Large Professeur Kali\\[0.3cm]
    \normalsize Base sur les challenges Advent of Code
}
\date{\today}

% === DOCUMENT ===
\begin{document}

\frontmatter
\maketitle

\tableofcontents

\chapter*{Preface}
\addcontentsline{toc}{chapter}{Preface}

Ce livre est ne d'une conviction : \textbf{on apprend mieux en resolvant des problemes qu'en lisant des cours}.

Les challenges d'Advent of Code, crees par Eric Wastl depuis 2015, offrent une progression ideale pour apprendre la programmation et la pensee algorithmique. Plus de 250 problemes, du plus simple au plus complexe, couvrant pratiquement tous les concepts fondamentaux de l'informatique.

\textbf{Comment utiliser ce livre ?}

\begin{itemize}
    \item Chaque chapitre couvre un \textbf{concept} (pas une annee d'AoC)
    \item Les concepts sont ordonnes par \textbf{prerequis}
    \item Chaque concept est illustre par plusieurs \textbf{challenges}
    \item Les explications vont du plus simple (10 ans) au plus technique
\end{itemize}

\begin{cyber}
Chaque concept est relie a une application concrete en cybersecurite. Tu ne fais pas que resoudre des puzzles - tu construis des competences reelles.
\end{cyber}

\vfill

\textit{``On n'apprend pas a nager en lisant un livre. On saute dans l'eau, on boit la tasse, puis on comprend.''}

\chapter*{Carte des Concepts}
\addcontentsline{toc}{chapter}{Carte des Concepts}

Le graphe ci-dessous montre les \textbf{dependances entre concepts}. Une fleche de A vers B signifie que tu dois maitriser A avant d'apprendre B.

\begin{center}
\resizebox{\textwidth}{!}{%
% Graphe de dependances des concepts - LearnByAoc
% A inclure dans le document principal avec % Graphe de dependances des concepts - LearnByAoc
% A inclure dans le document principal avec % Graphe de dependances des concepts - LearnByAoc
% A inclure dans le document principal avec \input{assets/dependency_graph.tex}

\begin{tikzpicture}[
    % Styles des noeuds par niveau
    debutant/.style={rectangle, rounded corners, draw=green!60!black, fill=green!20,
                     text width=2.2cm, align=center, minimum height=0.8cm, font=\small},
    intermediaire/.style={rectangle, rounded corners, draw=blue!60!black, fill=blue!20,
                          text width=2.2cm, align=center, minimum height=0.8cm, font=\small},
    avance/.style={rectangle, rounded corners, draw=red!60!black, fill=red!20,
                   text width=2.2cm, align=center, minimum height=0.8cm, font=\small},
    expert/.style={rectangle, rounded corners, draw=purple!60!black, fill=purple!20,
                   text width=2.2cm, align=center, minimum height=0.8cm, font=\small},
    % Style des fleches
    arrow/.style={->, >=stealth, thick, color=gray!70},
    % Style des labels de section
    section/.style={font=\bfseries\large, color=black!80}
]

% ==========================================
% SECTION 1: STRUCTURES DE DONNEES (gauche)
% ==========================================
\node[section] at (-6, 8) {Structures de Donnees};

% Niveau Debutant
\node[debutant] (list) at (-8, 6) {list\\(245 occ.)};
\node[debutant] (tuple) at (-6, 6) {tuple\\(69 occ.)};
\node[debutant] (dict) at (-4, 6) {dict\\(43 occ.)};

% Niveau Intermediaire
\node[intermediaire] (set) at (-8, 4) {set\\(324 occ.)};
\node[intermediaire] (defaultdict) at (-5, 4) {defaultdict\\(121 occ.)};
\node[intermediaire] (counter) at (-2, 4) {Counter\\(93 occ.)};

% Niveau Intermediaire+
\node[intermediaire] (deque) at (-7, 2) {deque\\(109 occ.)};
\node[intermediaire] (queue) at (-4, 2) {queue\\(211 occ.)};
\node[intermediaire] (stack) at (-1, 2) {stack\\(98 occ.)};

% Niveau Avance
\node[avance] (heap) at (-5.5, 0) {heap\\(3 occ.)};

% Fleches structures
\draw[arrow] (list) -- (set);
\draw[arrow] (dict) -- (defaultdict);
\draw[arrow] (dict) -- (counter);
\draw[arrow] (set) -- (deque);
\draw[arrow] (deque) -- (queue);
\draw[arrow] (list) -- (stack);
\draw[arrow] (queue) -- (heap);

% ==========================================
% SECTION 2: ALGORITHMES (centre)
% ==========================================
\node[section] at (3, 8) {Algorithmes};

% Niveau Debutant
\node[debutant] (boucles) at (1, 6) {Boucles};
\node[debutant] (brute) at (4, 6) {Brute Force\\(6 occ.)};

% Niveau Intermediaire
\node[intermediaire] (recursion) at (0, 4) {Recursion\\(6 occ.)};
\node[intermediaire] (tri) at (3, 4) {Tri};
\node[intermediaire] (greedy) at (6, 4) {Greedy\\(10 occ.)};

% Niveau Intermediaire+
\node[intermediaire] (bfs) at (-1, 2) {BFS\\(94 occ.)};
\node[intermediaire] (dfs) at (2, 2) {DFS\\(32 occ.)};
\node[intermediaire] (sliding) at (5, 2) {Sliding Window\\(1 occ.)};

% Niveau Avance
\node[avance] (dijkstra) at (-1, 0) {Dijkstra\\(17 occ.)};
\node[avance] (backtrack) at (2, 0) {Backtracking\\(6 occ.)};
\node[avance] (memo) at (5, 0) {Memoization\\(3 occ.)};

% Niveau Expert
\node[expert] (astar) at (-1, -2) {A*};
\node[expert] (dp) at (3.5, -2) {Prog. Dynamique};

% Fleches algorithmes
\draw[arrow] (boucles) -- (brute);
\draw[arrow] (boucles) -- (recursion);
\draw[arrow] (boucles) -- (tri);
\draw[arrow] (tri) -- (greedy);
\draw[arrow] (recursion) -- (dfs);
\draw[arrow] (recursion) -- (memo);
\draw[arrow] (dfs) -- (backtrack);
\draw[arrow] (memo) -- (dp);
\draw[arrow] (backtrack) -- (dp);

% Connexions inter-sections
\draw[arrow, dashed, color=orange] (queue) to[bend left=20] (bfs);
\draw[arrow, dashed, color=orange] (set) to[bend left=30] (bfs);
\draw[arrow] (bfs) -- (dijkstra);
\draw[arrow, dashed, color=orange] (heap) to[bend right=20] (dijkstra);
\draw[arrow] (dijkstra) -- (astar);

% ==========================================
% SECTION 3: MATHEMATIQUES (droite)
% ==========================================
\node[section] at (11, 8) {Mathematiques};

% Niveau Debutant
\node[debutant] (arith) at (9, 6) {Arithmetique};
\node[debutant] (distance) at (12, 6) {Distance\\(170 occ.)};

% Niveau Intermediaire
\node[intermediaire] (modulo) at (8, 4) {Modulo\\(75 occ.)};
\node[intermediaire] (gcd) at (11, 4) {GCD\\(15 occ.)};
\node[intermediaire] (manhattan) at (14, 4) {Manhattan\\(57 occ.)};

% Niveau Intermediaire+
\node[intermediaire] (lcm) at (9.5, 2) {LCM\\(10 occ.)};
\node[intermediaire] (binary) at (12.5, 2) {Binaire\\(41 occ.)};

% Niveau Avance
\node[avance] (xor) at (11, 0) {XOR\\(42 occ.)};
\node[avance] (bitwise) at (14, 0) {Bitwise\\(3 occ.)};

% Niveau Expert
\node[expert] (crt) at (9.5, -2) {Th. Chinois\\Restes};
\node[expert] (modular) at (13, -2) {Arith.\\Modulaire};

% Fleches maths
\draw[arrow] (arith) -- (modulo);
\draw[arrow] (arith) -- (gcd);
\draw[arrow] (distance) -- (manhattan);
\draw[arrow] (gcd) -- (lcm);
\draw[arrow] (modulo) -- (lcm);
\draw[arrow] (binary) -- (xor);
\draw[arrow] (xor) -- (bitwise);
\draw[arrow] (lcm) -- (crt);
\draw[arrow] (modulo) -- (modular);
\draw[arrow] (bitwise) -- (modular);

% ==========================================
% LEGENDE
% ==========================================
\node[section] at (3, -4.5) {Legende};
\node[debutant, text width=1.8cm] at (-2, -5.5) {Debutant};
\node[intermediaire, text width=2cm] at (1, -5.5) {Intermediaire};
\node[avance, text width=1.5cm] at (4, -5.5) {Avance};
\node[expert, text width=1.5cm] at (7, -5.5) {Expert};

\draw[arrow] (8, -5.5) -- (9.5, -5.5) node[right, font=\small] {Prerequis};
\draw[arrow, dashed, color=orange] (11, -5.5) -- (12.5, -5.5) node[right, font=\small] {Lien inter-domaine};

\end{tikzpicture}


\begin{tikzpicture}[
    % Styles des noeuds par niveau
    debutant/.style={rectangle, rounded corners, draw=green!60!black, fill=green!20,
                     text width=2.2cm, align=center, minimum height=0.8cm, font=\small},
    intermediaire/.style={rectangle, rounded corners, draw=blue!60!black, fill=blue!20,
                          text width=2.2cm, align=center, minimum height=0.8cm, font=\small},
    avance/.style={rectangle, rounded corners, draw=red!60!black, fill=red!20,
                   text width=2.2cm, align=center, minimum height=0.8cm, font=\small},
    expert/.style={rectangle, rounded corners, draw=purple!60!black, fill=purple!20,
                   text width=2.2cm, align=center, minimum height=0.8cm, font=\small},
    % Style des fleches
    arrow/.style={->, >=stealth, thick, color=gray!70},
    % Style des labels de section
    section/.style={font=\bfseries\large, color=black!80}
]

% ==========================================
% SECTION 1: STRUCTURES DE DONNEES (gauche)
% ==========================================
\node[section] at (-6, 8) {Structures de Donnees};

% Niveau Debutant
\node[debutant] (list) at (-8, 6) {list\\(245 occ.)};
\node[debutant] (tuple) at (-6, 6) {tuple\\(69 occ.)};
\node[debutant] (dict) at (-4, 6) {dict\\(43 occ.)};

% Niveau Intermediaire
\node[intermediaire] (set) at (-8, 4) {set\\(324 occ.)};
\node[intermediaire] (defaultdict) at (-5, 4) {defaultdict\\(121 occ.)};
\node[intermediaire] (counter) at (-2, 4) {Counter\\(93 occ.)};

% Niveau Intermediaire+
\node[intermediaire] (deque) at (-7, 2) {deque\\(109 occ.)};
\node[intermediaire] (queue) at (-4, 2) {queue\\(211 occ.)};
\node[intermediaire] (stack) at (-1, 2) {stack\\(98 occ.)};

% Niveau Avance
\node[avance] (heap) at (-5.5, 0) {heap\\(3 occ.)};

% Fleches structures
\draw[arrow] (list) -- (set);
\draw[arrow] (dict) -- (defaultdict);
\draw[arrow] (dict) -- (counter);
\draw[arrow] (set) -- (deque);
\draw[arrow] (deque) -- (queue);
\draw[arrow] (list) -- (stack);
\draw[arrow] (queue) -- (heap);

% ==========================================
% SECTION 2: ALGORITHMES (centre)
% ==========================================
\node[section] at (3, 8) {Algorithmes};

% Niveau Debutant
\node[debutant] (boucles) at (1, 6) {Boucles};
\node[debutant] (brute) at (4, 6) {Brute Force\\(6 occ.)};

% Niveau Intermediaire
\node[intermediaire] (recursion) at (0, 4) {Recursion\\(6 occ.)};
\node[intermediaire] (tri) at (3, 4) {Tri};
\node[intermediaire] (greedy) at (6, 4) {Greedy\\(10 occ.)};

% Niveau Intermediaire+
\node[intermediaire] (bfs) at (-1, 2) {BFS\\(94 occ.)};
\node[intermediaire] (dfs) at (2, 2) {DFS\\(32 occ.)};
\node[intermediaire] (sliding) at (5, 2) {Sliding Window\\(1 occ.)};

% Niveau Avance
\node[avance] (dijkstra) at (-1, 0) {Dijkstra\\(17 occ.)};
\node[avance] (backtrack) at (2, 0) {Backtracking\\(6 occ.)};
\node[avance] (memo) at (5, 0) {Memoization\\(3 occ.)};

% Niveau Expert
\node[expert] (astar) at (-1, -2) {A*};
\node[expert] (dp) at (3.5, -2) {Prog. Dynamique};

% Fleches algorithmes
\draw[arrow] (boucles) -- (brute);
\draw[arrow] (boucles) -- (recursion);
\draw[arrow] (boucles) -- (tri);
\draw[arrow] (tri) -- (greedy);
\draw[arrow] (recursion) -- (dfs);
\draw[arrow] (recursion) -- (memo);
\draw[arrow] (dfs) -- (backtrack);
\draw[arrow] (memo) -- (dp);
\draw[arrow] (backtrack) -- (dp);

% Connexions inter-sections
\draw[arrow, dashed, color=orange] (queue) to[bend left=20] (bfs);
\draw[arrow, dashed, color=orange] (set) to[bend left=30] (bfs);
\draw[arrow] (bfs) -- (dijkstra);
\draw[arrow, dashed, color=orange] (heap) to[bend right=20] (dijkstra);
\draw[arrow] (dijkstra) -- (astar);

% ==========================================
% SECTION 3: MATHEMATIQUES (droite)
% ==========================================
\node[section] at (11, 8) {Mathematiques};

% Niveau Debutant
\node[debutant] (arith) at (9, 6) {Arithmetique};
\node[debutant] (distance) at (12, 6) {Distance\\(170 occ.)};

% Niveau Intermediaire
\node[intermediaire] (modulo) at (8, 4) {Modulo\\(75 occ.)};
\node[intermediaire] (gcd) at (11, 4) {GCD\\(15 occ.)};
\node[intermediaire] (manhattan) at (14, 4) {Manhattan\\(57 occ.)};

% Niveau Intermediaire+
\node[intermediaire] (lcm) at (9.5, 2) {LCM\\(10 occ.)};
\node[intermediaire] (binary) at (12.5, 2) {Binaire\\(41 occ.)};

% Niveau Avance
\node[avance] (xor) at (11, 0) {XOR\\(42 occ.)};
\node[avance] (bitwise) at (14, 0) {Bitwise\\(3 occ.)};

% Niveau Expert
\node[expert] (crt) at (9.5, -2) {Th. Chinois\\Restes};
\node[expert] (modular) at (13, -2) {Arith.\\Modulaire};

% Fleches maths
\draw[arrow] (arith) -- (modulo);
\draw[arrow] (arith) -- (gcd);
\draw[arrow] (distance) -- (manhattan);
\draw[arrow] (gcd) -- (lcm);
\draw[arrow] (modulo) -- (lcm);
\draw[arrow] (binary) -- (xor);
\draw[arrow] (xor) -- (bitwise);
\draw[arrow] (lcm) -- (crt);
\draw[arrow] (modulo) -- (modular);
\draw[arrow] (bitwise) -- (modular);

% ==========================================
% LEGENDE
% ==========================================
\node[section] at (3, -4.5) {Legende};
\node[debutant, text width=1.8cm] at (-2, -5.5) {Debutant};
\node[intermediaire, text width=2cm] at (1, -5.5) {Intermediaire};
\node[avance, text width=1.5cm] at (4, -5.5) {Avance};
\node[expert, text width=1.5cm] at (7, -5.5) {Expert};

\draw[arrow] (8, -5.5) -- (9.5, -5.5) node[right, font=\small] {Prerequis};
\draw[arrow, dashed, color=orange] (11, -5.5) -- (12.5, -5.5) node[right, font=\small] {Lien inter-domaine};

\end{tikzpicture}


\begin{tikzpicture}[
    % Styles des noeuds par niveau
    debutant/.style={rectangle, rounded corners, draw=green!60!black, fill=green!20,
                     text width=2.2cm, align=center, minimum height=0.8cm, font=\small},
    intermediaire/.style={rectangle, rounded corners, draw=blue!60!black, fill=blue!20,
                          text width=2.2cm, align=center, minimum height=0.8cm, font=\small},
    avance/.style={rectangle, rounded corners, draw=red!60!black, fill=red!20,
                   text width=2.2cm, align=center, minimum height=0.8cm, font=\small},
    expert/.style={rectangle, rounded corners, draw=purple!60!black, fill=purple!20,
                   text width=2.2cm, align=center, minimum height=0.8cm, font=\small},
    % Style des fleches
    arrow/.style={->, >=stealth, thick, color=gray!70},
    % Style des labels de section
    section/.style={font=\bfseries\large, color=black!80}
]

% ==========================================
% SECTION 1: STRUCTURES DE DONNEES (gauche)
% ==========================================
\node[section] at (-6, 8) {Structures de Donnees};

% Niveau Debutant
\node[debutant] (list) at (-8, 6) {list\\(245 occ.)};
\node[debutant] (tuple) at (-6, 6) {tuple\\(69 occ.)};
\node[debutant] (dict) at (-4, 6) {dict\\(43 occ.)};

% Niveau Intermediaire
\node[intermediaire] (set) at (-8, 4) {set\\(324 occ.)};
\node[intermediaire] (defaultdict) at (-5, 4) {defaultdict\\(121 occ.)};
\node[intermediaire] (counter) at (-2, 4) {Counter\\(93 occ.)};

% Niveau Intermediaire+
\node[intermediaire] (deque) at (-7, 2) {deque\\(109 occ.)};
\node[intermediaire] (queue) at (-4, 2) {queue\\(211 occ.)};
\node[intermediaire] (stack) at (-1, 2) {stack\\(98 occ.)};

% Niveau Avance
\node[avance] (heap) at (-5.5, 0) {heap\\(3 occ.)};

% Fleches structures
\draw[arrow] (list) -- (set);
\draw[arrow] (dict) -- (defaultdict);
\draw[arrow] (dict) -- (counter);
\draw[arrow] (set) -- (deque);
\draw[arrow] (deque) -- (queue);
\draw[arrow] (list) -- (stack);
\draw[arrow] (queue) -- (heap);

% ==========================================
% SECTION 2: ALGORITHMES (centre)
% ==========================================
\node[section] at (3, 8) {Algorithmes};

% Niveau Debutant
\node[debutant] (boucles) at (1, 6) {Boucles};
\node[debutant] (brute) at (4, 6) {Brute Force\\(6 occ.)};

% Niveau Intermediaire
\node[intermediaire] (recursion) at (0, 4) {Recursion\\(6 occ.)};
\node[intermediaire] (tri) at (3, 4) {Tri};
\node[intermediaire] (greedy) at (6, 4) {Greedy\\(10 occ.)};

% Niveau Intermediaire+
\node[intermediaire] (bfs) at (-1, 2) {BFS\\(94 occ.)};
\node[intermediaire] (dfs) at (2, 2) {DFS\\(32 occ.)};
\node[intermediaire] (sliding) at (5, 2) {Sliding Window\\(1 occ.)};

% Niveau Avance
\node[avance] (dijkstra) at (-1, 0) {Dijkstra\\(17 occ.)};
\node[avance] (backtrack) at (2, 0) {Backtracking\\(6 occ.)};
\node[avance] (memo) at (5, 0) {Memoization\\(3 occ.)};

% Niveau Expert
\node[expert] (astar) at (-1, -2) {A*};
\node[expert] (dp) at (3.5, -2) {Prog. Dynamique};

% Fleches algorithmes
\draw[arrow] (boucles) -- (brute);
\draw[arrow] (boucles) -- (recursion);
\draw[arrow] (boucles) -- (tri);
\draw[arrow] (tri) -- (greedy);
\draw[arrow] (recursion) -- (dfs);
\draw[arrow] (recursion) -- (memo);
\draw[arrow] (dfs) -- (backtrack);
\draw[arrow] (memo) -- (dp);
\draw[arrow] (backtrack) -- (dp);

% Connexions inter-sections
\draw[arrow, dashed, color=orange] (queue) to[bend left=20] (bfs);
\draw[arrow, dashed, color=orange] (set) to[bend left=30] (bfs);
\draw[arrow] (bfs) -- (dijkstra);
\draw[arrow, dashed, color=orange] (heap) to[bend right=20] (dijkstra);
\draw[arrow] (dijkstra) -- (astar);

% ==========================================
% SECTION 3: MATHEMATIQUES (droite)
% ==========================================
\node[section] at (11, 8) {Mathematiques};

% Niveau Debutant
\node[debutant] (arith) at (9, 6) {Arithmetique};
\node[debutant] (distance) at (12, 6) {Distance\\(170 occ.)};

% Niveau Intermediaire
\node[intermediaire] (modulo) at (8, 4) {Modulo\\(75 occ.)};
\node[intermediaire] (gcd) at (11, 4) {GCD\\(15 occ.)};
\node[intermediaire] (manhattan) at (14, 4) {Manhattan\\(57 occ.)};

% Niveau Intermediaire+
\node[intermediaire] (lcm) at (9.5, 2) {LCM\\(10 occ.)};
\node[intermediaire] (binary) at (12.5, 2) {Binaire\\(41 occ.)};

% Niveau Avance
\node[avance] (xor) at (11, 0) {XOR\\(42 occ.)};
\node[avance] (bitwise) at (14, 0) {Bitwise\\(3 occ.)};

% Niveau Expert
\node[expert] (crt) at (9.5, -2) {Th. Chinois\\Restes};
\node[expert] (modular) at (13, -2) {Arith.\\Modulaire};

% Fleches maths
\draw[arrow] (arith) -- (modulo);
\draw[arrow] (arith) -- (gcd);
\draw[arrow] (distance) -- (manhattan);
\draw[arrow] (gcd) -- (lcm);
\draw[arrow] (modulo) -- (lcm);
\draw[arrow] (binary) -- (xor);
\draw[arrow] (xor) -- (bitwise);
\draw[arrow] (lcm) -- (crt);
\draw[arrow] (modulo) -- (modular);
\draw[arrow] (bitwise) -- (modular);

% ==========================================
% LEGENDE
% ==========================================
\node[section] at (3, -4.5) {Legende};
\node[debutant, text width=1.8cm] at (-2, -5.5) {Debutant};
\node[intermediaire, text width=2cm] at (1, -5.5) {Intermediaire};
\node[avance, text width=1.5cm] at (4, -5.5) {Avance};
\node[expert, text width=1.5cm] at (7, -5.5) {Expert};

\draw[arrow] (8, -5.5) -- (9.5, -5.5) node[right, font=\small] {Prerequis};
\draw[arrow, dashed, color=orange] (11, -5.5) -- (12.5, -5.5) node[right, font=\small] {Lien inter-domaine};

\end{tikzpicture}

}
\end{center}

\textbf{Legende des couleurs :}
\begin{itemize}
    \item \textcolor{green!60!black}{\textbf{Vert}} : Niveau Debutant (Lycee)
    \item \textcolor{blue!60!black}{\textbf{Bleu}} : Niveau Intermediaire (Bac / L1)
    \item \textcolor{red!60!black}{\textbf{Rouge}} : Niveau Avance (L2 / L3)
    \item \textcolor{purple!60!black}{\textbf{Violet}} : Niveau Expert (Master)
\end{itemize}

Les nombres entre parentheses indiquent combien de fois ce concept apparait dans les 257 challenges Advent of Code (2015-2025).

\mainmatter

% === PARTIE 1 : FONDATIONS ===
\part{Fondations}

\chapter{Parsing : Lire et Transformer les Donnees}
\label{chap:parsing}

\begin{histoire}
Imagine que tu recois une lettre codee. Avant de pouvoir comprendre le message, tu dois d'abord \textbf{decoder} les symboles. C'est exactement ce qu'on fait en programmation : transformer du texte brut en donnees utilisables.
\end{histoire}

% ============================================================
\section{Qu'est-ce que le parsing ?}
% ============================================================

\begin{definition}[Parsing]
Le \textbf{parsing} (ou analyse syntaxique) est l'action de lire des donnees brutes (texte, fichier) et de les transformer en une structure exploitable par un programme.
\end{definition}

\begin{exemple}
Transformer la chaine \texttt{"123,456,789"} en liste \texttt{[123, 456, 789]}.
\end{exemple}

% ============================================================
\section{Lire un fichier}
% ============================================================

\subsection{La methode de base}

\begin{lstlisting}[caption={Lire un fichier ligne par ligne}]
# Methode 1 : Lire tout le fichier
with open("input.txt") as f:
    contenu = f.read()

# Methode 2 : Lire ligne par ligne
with open("input.txt") as f:
    lignes = f.readlines()

# Methode 3 : Iteration directe (recommande)
with open("input.txt") as f:
    for ligne in f:
        print(ligne.strip())
\end{lstlisting}

\begin{concept}
Le mot-cle \texttt{with} garantit que le fichier sera ferme automatiquement, meme en cas d'erreur. C'est une bonne pratique a toujours utiliser.
\end{concept}

\subsection{Nettoyer les donnees}

\begin{lstlisting}[caption={Methodes de nettoyage}]
ligne = "  Hello World  \n"

ligne.strip()      # "Hello World" (supprime espaces + \n)
ligne.rstrip()     # "  Hello World" (seulement a droite)
ligne.lstrip()     # "Hello World  \n" (seulement a gauche)
\end{lstlisting}

% ============================================================
\section{Decouper une chaine}
% ============================================================

\subsection{split() : le couteau suisse}

\begin{lstlisting}[caption={Utilisation de split()}]
# Decouper par espaces (defaut)
"a b c".split()        # ['a', 'b', 'c']

# Decouper par un caractere specifique
"a,b,c".split(",")     # ['a', 'b', 'c']

# Decouper par une chaine
"a->b->c".split("->")  # ['a', 'b', 'c']

# Limiter le nombre de decoupages
"a,b,c,d".split(",", 2)  # ['a', 'b', 'c,d']
\end{lstlisting}

\begin{piege}
\texttt{split()} sans argument decoupe par \textbf{tous les espaces blancs} (espaces, tabs, newlines) et supprime les elements vides. \texttt{split(" ")} ne decoupe que par l'espace simple.

\begin{lstlisting}
"a  b".split()     # ['a', 'b'] (2 espaces ignores)
"a  b".split(" ")  # ['a', '', 'b'] (element vide!)
\end{lstlisting}
\end{piege}

% ============================================================
\section{Convertir les types}
% ============================================================

\begin{lstlisting}[caption={Conversions de types}]
# String vers entier
int("42")          # 42
int("  42  ")      # 42 (strip automatique)
int("42abc")       # ERREUR !

# String vers flottant
float("3.14")      # 3.14

# Entier vers string
str(42)            # "42"

# Conversion en masse avec map()
nombres = list(map(int, ["1", "2", "3"]))  # [1, 2, 3]

# Ou avec une comprehension (plus lisible)
nombres = [int(x) for x in ["1", "2", "3"]]  # [1, 2, 3]
\end{lstlisting}

% ============================================================
\section{Patterns de parsing courants}
% ============================================================

\subsection{Pattern 1 : Liste de nombres}

\textbf{Entree :}
\begin{verbatim}
123
456
789
\end{verbatim}

\begin{lstlisting}[caption={Parser une liste de nombres}]
with open("input.txt") as f:
    nombres = [int(ligne) for ligne in f]
# [123, 456, 789]
\end{lstlisting}

\subsection{Pattern 2 : Valeurs separees par virgules}

\textbf{Entree :} \texttt{1,2,3,4,5}

\begin{lstlisting}[caption={Parser des valeurs CSV}]
with open("input.txt") as f:
    nombres = [int(x) for x in f.read().strip().split(",")]
# [1, 2, 3, 4, 5]
\end{lstlisting}

\subsection{Pattern 3 : Grille 2D}

\textbf{Entree :}
\begin{verbatim}
@.@.
.@@.
@.@.
\end{verbatim}

\begin{lstlisting}[caption={Parser une grille}]
with open("input.txt") as f:
    grille = [list(ligne.strip()) for ligne in f]
# [['@', '.', '@', '.'],
#  ['.', '@', '@', '.'],
#  ['@', '.', '@', '.']]

# Acces : grille[ligne][colonne]
grille[0][0]  # '@'
grille[1][1]  # '@'
\end{lstlisting}

\subsection{Pattern 4 : Sections separees par ligne vide}

\textbf{Entree :}
\begin{verbatim}
section1
data1

section2
data2
\end{verbatim}

\begin{lstlisting}[caption={Parser des sections}]
with open("input.txt") as f:
    sections = f.read().strip().split("\n\n")
# ['section1\ndata1', 'section2\ndata2']

# Puis parser chaque section
for section in sections:
    lignes = section.split("\n")
    # ...
\end{lstlisting}

% ============================================================
\section{Challenges d'application}
% ============================================================

\begin{exercice}
\textbf{AoC 2015 Day 1 - Not Quite Lisp}

Parse une chaine de caracteres \texttt{(} et \texttt{)} et compte combien de fois chaque caractere apparait.

\textbf{Difficulte :} \faStar

\textbf{Concepts :} Iteration sur string, comptage
\end{exercice}

\begin{exercice}
\textbf{AoC 2020 Day 4 - Passport Processing}

Parse des ``passeports'' avec des champs \texttt{cle:valeur} separes par espaces ou newlines.

\textbf{Difficulte :} \faStar\faStar

\textbf{Concepts :} Split multiple, dictionnaires
\end{exercice}

\begin{exercice}
\textbf{AoC 2016 Day 4 - Security Through Obscurity}

Parse des noms de salles au format \texttt{nom-avec-tirets-123[checksum]}.

\textbf{Difficulte :} \faStar\faStar\faStar

\textbf{Concepts :} Regex, extraction de patterns
\end{exercice}

% Lien interactif vers Google Colab
\colab{Notebook interactif - Parsing}{https://colab.research.google.com/github/kless/LearnByAoC/blob/main/notebooks/01_parsing.ipynb}

% ============================================================
\section{Application en cybersecurite}
% ============================================================

\begin{cyber}
Le parsing est \textbf{fondamental} en cyber :

\begin{itemize}
    \item \textbf{Analyse de logs} : Parser les logs Apache, auth.log, syslog
    \item \textbf{Parsing de packets} : Extraire les headers HTTP, DNS queries
    \item \textbf{Reverse engineering} : Parser des formats binaires
    \item \textbf{OSINT} : Extraire des donnees de pages web
\end{itemize}

\textbf{Attention :} Un parsing mal fait peut creer des vulnerabilites (injection, buffer overflow). Toujours valider les donnees !
\end{cyber}

% ============================================================
\section{Resume}
% ============================================================

\begin{tabular}{ll}
\toprule
\textbf{Methode} & \textbf{Usage} \\
\midrule
\texttt{open() / with} & Lire un fichier \\
\texttt{.strip()} & Nettoyer les espaces \\
\texttt{.split()} & Decouper une chaine \\
\texttt{int() / float()} & Convertir en nombre \\
\texttt{map()} & Appliquer une fonction \\
\texttt{[x for x in ...]} & Comprehension de liste \\
\bottomrule
\end{tabular}

\chapter{Arithmetique : Compter et Calculer}
\label{chap:arithmetique}

\begin{histoire}
Tu te souviens quand tu as appris a compter sur tes doigts ? En programmation, on fait pareil - mais avec des millions de nombres, et beaucoup plus vite !
\end{histoire}

% ============================================================
\section{Les operations de base}
% ============================================================

\begin{definition}[Operateurs arithmetiques]
Python utilise les operateurs classiques :
\begin{itemize}
    \item \texttt{+} : Addition
    \item \texttt{-} : Soustraction
    \item \texttt{*} : Multiplication
    \item \texttt{/} : Division (resultat flottant)
    \item \texttt{//} : Division entiere
    \item \texttt{\%} : Modulo (reste de la division)
    \item \texttt{**} : Puissance
\end{itemize}
\end{definition}

\begin{lstlisting}[caption={Operations arithmetiques}]
17 / 5    # 3.4   (division flottante)
17 // 5   # 3     (division entiere)
17 % 5    # 2     (reste : 17 = 3*5 + 2)
2 ** 10   # 1024  (2 puissance 10)
\end{lstlisting}

% ============================================================
\section{Le modulo : ton nouvel ami}
% ============================================================

\begin{concept}
Le \textbf{modulo} (\texttt{\%}) donne le \textbf{reste} de la division. C'est l'operation la plus utile en programmation !
\end{concept}

\begin{lstlisting}[caption={Usages courants du modulo}]
# Pair ou impair ?
n % 2 == 0  # True si n est pair

# Position dans un cycle (0 a n-1)
jour = total_jours % 7  # Jour de la semaine (0-6)

# Eviter les depassements de liste
index = i % len(liste)  # Revient au debut si depasse

# Extraire le dernier chiffre
dernier = n % 10  # Chiffre des unites
\end{lstlisting}

\begin{exercice}
\textbf{AoC 2015 Day 1 - Not Quite Lisp}

Utilise un compteur simple : \texttt{(} ajoute 1, \texttt{)} soustrait 1. Trouve l'etage final.

\textbf{Concepts :} Iteration, comptage, conditions
\end{exercice}

% ============================================================
\section{GCD et LCM : cycles et periodicite}
% ============================================================

\begin{definition}[PGCD et PPCM]
\begin{itemize}
    \item \textbf{PGCD} (GCD) : Plus Grand Commun Diviseur
    \item \textbf{PPCM} (LCM) : Plus Petit Commun Multiple
\end{itemize}
\end{definition}

\begin{lstlisting}[caption={Calcul du LCM}]
import math

def lcm(a, b):
    return a * b // math.gcd(a, b)

# Quand 3 cycles de periodes 4, 6, 9 se synchronisent ?
from functools import reduce
periodes = [4, 6, 9]
sync = reduce(lcm, periodes)  # 36
\end{lstlisting}

\begin{cyber}
Le PGCD est utilise en \textbf{cryptographie} (RSA) et le PPCM pour analyser les \textbf{cycles} dans les malwares.
\end{cyber}

% ============================================================
\section{Challenges d'application}
% ============================================================

\begin{exercice}
\textbf{AoC 2020 Day 13 - Shuttle Search}

Trouve quand plusieurs bus avec des periodes differentes passent en meme temps.

\textbf{Difficulte :} \faStar\faStar\faStar\faStar

\textbf{Concepts :} LCM, modulo, arithmetique modulaire
\end{exercice}

% ============================================================
\section{Resume}
% ============================================================

\begin{tabular}{ll}
\toprule
\textbf{Concept} & \textbf{Usage principal} \\
\midrule
\texttt{\%} (modulo) & Cycles, parite, position circulaire \\
\texttt{//} (div. entiere) & Quotient sans decimales \\
\texttt{math.gcd()} & Plus grand diviseur commun \\
\texttt{math.lcm()} & Synchronisation de cycles \\
\bottomrule
\end{tabular}

\chapter{Recherche : Trouver ce qu'on Cherche}
\label{chap:recherche}

\begin{histoire}
Imagine que tu cherches un mot dans le dictionnaire. Tu ne lis pas toutes les pages depuis le debut ! Tu ouvres au milieu, tu regardes si c'est avant ou apres, et tu recommences. C'est la \textbf{recherche binaire}.
\end{histoire}

% ============================================================
\section{Recherche lineaire}
% ============================================================

\begin{definition}[Recherche lineaire]
Parcourir tous les elements un par un jusqu'a trouver ce qu'on cherche. Complexite : O(n).
\end{definition}

% ============================================================
\section{Recherche dans un set : O(1)}
% ============================================================

\begin{concept}
Un \texttt{set} utilise une \textbf{table de hachage}. La recherche est instantanee (O(1)).
\end{concept}

\begin{lstlisting}[caption={Comparaison list vs set}]
# LENT : recherche dans une liste O(n)
if x in liste:  # parcourt toute la liste
    ...

# RAPIDE : recherche dans un set O(1)
ensemble = set(liste)
if x in ensemble:  # acces direct
    ...
\end{lstlisting}

% ============================================================
\section{Recherche binaire}
% ============================================================

\begin{definition}[Recherche binaire]
Diviser l'espace de recherche en deux a chaque etape. Complexite : O(log n).
\end{definition}

\begin{lstlisting}[caption={Module bisect}]
import bisect

liste = [1, 3, 5, 7, 9]
bisect.bisect_left(liste, 6)   # 3 (position d'insertion)
\end{lstlisting}

% ============================================================
\section{Challenges d'application}
% ============================================================

\begin{exercice}
\textbf{AoC 2020 Day 1 - Report Repair}

Trouve deux nombres dont la somme est 2020. Optimise avec un set.

\textbf{Concepts :} Recherche dans set, complement
\end{exercice}

\begin{cyber}
\begin{itemize}
    \item \textbf{Bruteforce intelligent} : Recherche binaire sur l'espace de cles
    \item \textbf{Analyse de logs} : Recherche rapide dans des GB de logs
\end{itemize}
\end{cyber}

% ============================================================
\section{Resume}
% ============================================================

\begin{tabular}{lll}
\toprule
\textbf{Methode} & \textbf{Complexite} & \textbf{Quand} \\
\midrule
Liste + \texttt{in} & O(n) & Petite liste \\
Set + \texttt{in} & O(1) & Recherches multiples \\
Recherche binaire & O(log n) & Donnees triees \\
\bottomrule
\end{tabular}

\chapter{Grilles 2D : Naviguer dans l'Espace}
\label{chap:grilles}

\begin{histoire}
Tu connais les echecs ? Une grille 8x8 ou chaque piece peut bouger dans certaines directions. En programmation, on utilise constamment des grilles !
\end{histoire}

% ============================================================
\section{Representer une grille}
% ============================================================

\begin{lstlisting}[caption={Creer et parser une grille}]
# Parser un fichier en grille
with open("input.txt") as f:
    grille = [list(ligne.strip()) for ligne in f]

# Acces : grille[ligne][colonne]
grille[0][0]  # coin haut-gauche

# Dimensions
hauteur = len(grille)
largeur = len(grille[0])
\end{lstlisting}

\begin{piege}
Attention : \texttt{grille[ligne][colonne]} = \texttt{grille[y][x]} !
\end{piege}

% ============================================================
\section{Les 4 directions}
% ============================================================

\begin{lstlisting}[caption={Directions cardinales}]
DIRECTIONS = [
    (-1, 0),  # Haut
    (1, 0),   # Bas
    (0, -1),  # Gauche
    (0, 1)    # Droite
]
\end{lstlisting}

% ============================================================
\section{Distance de Manhattan}
% ============================================================

\begin{definition}[Distance de Manhattan]
Distance en blocs : $d = |x_1 - x_2| + |y_1 - y_2|$
\end{definition}

\begin{lstlisting}[caption={Calcul Manhattan}]
def manhattan(p1, p2):
    return abs(p1[0] - p2[0]) + abs(p1[1] - p2[1])
\end{lstlisting}

% ============================================================
\section{Challenges d'application}
% ============================================================

\begin{exercice}
\textbf{AoC 2015 Day 3 - Perfectly Spherical Houses}

Suis des directions et compte les maisons visitees.

\textbf{Concepts :} Directions, set de positions
\end{exercice}

\begin{cyber}
\begin{itemize}
    \item \textbf{Analyse d'images} : Steganographie, detection
    \item \textbf{Cartographie reseau} : Visualisation de topologie
\end{itemize}
\end{cyber}


% === PARTIE 2 : STRUCTURES ===
\part{Structures de Donnees}

\chapter{Ensembles (Sets) : L'Unicite}
\label{chap:ensembles}

\begin{histoire}
Imagine un sac magique ou tu ne peux mettre qu'un seul exemplaire de chaque objet. Si tu essaies de mettre deux pommes identiques, le sac n'en garde qu'une. C'est un \textbf{ensemble} !
\end{histoire}

% ============================================================
\section{Qu'est-ce qu'un set ?}
% ============================================================

\begin{definition}[Set]
Un \texttt{set} est une collection \textbf{non ordonnee} d'elements \textbf{uniques}. Recherche en O(1).
\end{definition}

\begin{lstlisting}[caption={Operations de base}]
# Creation
ensemble = {1, 2, 3}
ensemble = set([1, 2, 2, 3])  # {1, 2, 3}

# Ajout / Suppression
ensemble.add(4)
ensemble.remove(1)      # Erreur si absent
ensemble.discard(10)    # Pas d'erreur si absent

# Test d'appartenance O(1)
if x in ensemble:
    ...
\end{lstlisting}

% ============================================================
\section{Operations ensemblistes}
% ============================================================

\begin{lstlisting}[caption={Union, intersection, difference}]
a = {1, 2, 3}
b = {2, 3, 4}

a | b   # {1, 2, 3, 4}  Union
a & b   # {2, 3}        Intersection
a - b   # {1}           Difference
a ^ b   # {1, 4}        Difference symetrique
\end{lstlisting}

% ============================================================
\section{Pattern : Positions visitees}
% ============================================================

\begin{lstlisting}[caption={Tracker les positions}]
visited = set()
position = (0, 0)

while True:
    if position in visited:
        print("Deja visite !")
        break
    visited.add(position)
    position = move(position)
\end{lstlisting}

% ============================================================
\section{Challenges d'application}
% ============================================================

\begin{exercice}
\textbf{AoC 2015 Day 3 - Perfectly Spherical Houses}

Compte les maisons uniques visitees par le Pere Noel.

\textbf{Concepts :} Set de tuples, positions
\end{exercice}

\begin{cyber}
\begin{itemize}
    \item \textbf{Deduplication} : Supprimer les doublons dans les logs
    \item \textbf{Detection d'anomalies} : Nouvelles IPs, nouveaux users
\end{itemize}
\end{cyber}

\chapter{Dictionnaires : Associer Cles et Valeurs}
\label{chap:dictionnaires}

\begin{histoire}
Un dictionnaire, c'est comme ton carnet de contacts : tu cherches un nom (la cle) et tu trouves le numero (la valeur). Instantanement.
\end{histoire}

% ============================================================
\section{Qu'est-ce qu'un dict ?}
% ============================================================

\begin{definition}[Dictionnaire]
Un \texttt{dict} associe des \textbf{cles} a des \textbf{valeurs}. Acces en O(1).
\end{definition}

\begin{lstlisting}[caption={Operations de base}]
# Creation
scores = {"Alice": 100, "Bob": 85}

# Acces
scores["Alice"]      # 100
scores.get("Eve", 0) # 0 (defaut si absent)

# Modification
scores["Alice"] = 110
scores["Eve"] = 95   # Nouvelle entree

# Iteration
for nom, score in scores.items():
    print(f"{nom}: {score}")
\end{lstlisting}

% ============================================================
\section{defaultdict : le dict intelligent}
% ============================================================

\begin{lstlisting}[caption={defaultdict pour eviter les KeyError}]
from collections import defaultdict

# Comptage
compteur = defaultdict(int)
for mot in mots:
    compteur[mot] += 1  # Pas besoin d'initialiser !

# Listes de valeurs
graphe = defaultdict(list)
graphe["A"].append("B")  # Pas besoin de verifier si "A" existe
\end{lstlisting}

% ============================================================
\section{Counter : compter automatiquement}
% ============================================================

\begin{lstlisting}[caption={Counter pour les frequences}]
from collections import Counter

texte = "abracadabra"
freq = Counter(texte)
# Counter({'a': 5, 'b': 2, 'r': 2, 'c': 1, 'd': 1})

freq.most_common(3)  # [('a', 5), ('b', 2), ('r', 2)]
\end{lstlisting}

% ============================================================
\section{Challenges d'application}
% ============================================================

\begin{exercice}
\textbf{AoC 2024 Day 1 - Historian Hysteria}

Compte les occurrences et calcule un score de similarite.

\textbf{Concepts :} Counter, multiplication
\end{exercice}

\begin{cyber}
\begin{itemize}
    \item \textbf{Analyse de frequence} : Casser le chiffrement par substitution
    \item \textbf{Cache/Memoization} : Accelerer les calculs
\end{itemize}
\end{cyber}

\chapter{Piles et Files : L'Ordre de Traitement}
\label{chap:piles_files}

\begin{histoire}
Une pile d'assiettes : tu poses en haut, tu prends en haut (LIFO). Une file d'attente au cinema : le premier arrive est le premier servi (FIFO).
\end{histoire}

% ============================================================
\section{La Pile (Stack) : LIFO}
% ============================================================

\begin{definition}[Pile]
Last In, First Out. Le dernier element ajoute est le premier retire.
\end{definition}

\begin{lstlisting}[caption={Pile avec une liste}]
pile = []
pile.append(1)  # Push
pile.append(2)
pile.pop()      # 2 (Pop)
pile[-1]        # 1 (Peek sans retirer)
\end{lstlisting}

% ============================================================
\section{La File (Queue) : FIFO}
% ============================================================

\begin{definition}[File]
First In, First Out. Le premier element ajoute est le premier retire.
\end{definition}

\begin{lstlisting}[caption={File avec deque}]
from collections import deque

file = deque()
file.append(1)     # Ajouter a droite
file.append(2)
file.popleft()     # 1 (Retirer a gauche)
\end{lstlisting}

\begin{concept}
\texttt{deque} est O(1) aux deux extremites. Une liste est O(n) pour \texttt{pop(0)} !
\end{concept}

% ============================================================
\section{Applications}
% ============================================================

\begin{itemize}
    \item \textbf{Pile} : Parentheses equilibrees, DFS, undo/redo
    \item \textbf{File} : BFS, traitement dans l'ordre d'arrivee
\end{itemize}

% ============================================================
\section{Challenges d'application}
% ============================================================

\begin{exercice}
\textbf{AoC 2021 Day 10 - Syntax Scoring}

Verifie l'equilibrage des parentheses avec une pile.

\textbf{Concepts :} Pile, matching de caracteres
\end{exercice}

\begin{cyber}
\begin{itemize}
    \item \textbf{Buffer overflow} : Comprendre la pile d'execution
    \item \textbf{Message queues} : Communication inter-processus
\end{itemize}
\end{cyber}

\chapter{Arbres : Structures Hierarchiques}
\label{chap:arbres}

\begin{histoire}
Un arbre genealogique : grand-parents en haut, parents au milieu, enfants en bas. Chaque personne a un parent (sauf la racine) et peut avoir plusieurs enfants.
\end{histoire}

% ============================================================
\section{Qu'est-ce qu'un arbre ?}
% ============================================================

\begin{definition}[Arbre]
Structure hierarchique avec une \textbf{racine}, des \textbf{noeuds} et des \textbf{feuilles} (noeuds sans enfants).
\end{definition}

\begin{lstlisting}[caption={Representation avec dict}]
# Arbre comme dict parent -> enfants
arbre = {
    "COM": ["B"],
    "B": ["C", "G"],
    "C": ["D"],
    "G": ["H"],
}

# Ou enfant -> parent (pour remonter)
parents = {
    "B": "COM",
    "C": "B",
    "D": "C",
}
\end{lstlisting}

% ============================================================
\section{Parcours d'arbre}
% ============================================================

\begin{lstlisting}[caption={Parcours recursif}]
def parcours(noeud, arbre, profondeur=0):
    print("  " * profondeur + noeud)
    for enfant in arbre.get(noeud, []):
        parcours(enfant, arbre, profondeur + 1)
\end{lstlisting}

% ============================================================
\section{Challenges d'application}
% ============================================================

\begin{exercice}
\textbf{AoC 2019 Day 6 - Universal Orbit Map}

Compte les orbites directes et indirectes dans un arbre.

\textbf{Concepts :} Parcours d'arbre, comptage de profondeur
\end{exercice}

\begin{cyber}
\begin{itemize}
    \item \textbf{Systeme de fichiers} : Arborescence de repertoires
    \item \textbf{DOM/XML} : Parsing de documents structures
\end{itemize}
\end{cyber}


% === PARTIE 3 : ALGORITHMES ===
\part{Algorithmes}

\chapter{Recursion : Se Rappeler Soi-meme}
\label{chap:recursion}

\begin{histoire}
Pour comprendre la recursion, il faut d'abord comprendre la recursion. C'est une blague de programmeur ! Une fonction qui s'appelle elle-meme pour resoudre un probleme plus petit.
\end{histoire}

% ============================================================
\section{Qu'est-ce que la recursion ?}
% ============================================================

\begin{definition}[Recursion]
Une fonction qui s'appelle elle-meme avec un cas de base pour s'arreter.
\end{definition}

\begin{lstlisting}[caption={Exemple : factorielle}]
def factorielle(n):
    if n <= 1:         # Cas de base
        return 1
    return n * factorielle(n - 1)  # Appel recursif

# 5! = 5 * 4 * 3 * 2 * 1 = 120
\end{lstlisting}

% ============================================================
\section{Memoization : eviter les recalculs}
% ============================================================

\begin{lstlisting}[caption={Fibonacci avec memoization}]
from functools import lru_cache

@lru_cache(maxsize=None)
def fib(n):
    if n < 2:
        return n
    return fib(n-1) + fib(n-2)

# Sans cache : O(2^n) - EXPLOSIF
# Avec cache : O(n) - lineaire
\end{lstlisting}

\begin{concept}
La \textbf{memoization} stocke les resultats deja calcules. Essentiel pour la programmation dynamique !
\end{concept}

% ============================================================
\section{Challenges d'application}
% ============================================================

\begin{exercice}
\textbf{AoC 2015 Day 7 - Some Assembly Required}

Evalue des circuits logiques avec des dependances recursives.

\textbf{Concepts :} Recursion, memoization, graphe de dependances
\end{exercice}

\begin{cyber}
\begin{itemize}
    \item \textbf{Parsing recursif} : Grammaires, expressions
    \item \textbf{Exploration de systemes} : Fichiers, processus
\end{itemize}
\end{cyber}

\chapter{Graphes : BFS, DFS et Dijkstra}
\label{chap:graphes}

\begin{histoire}
Un reseau social est un graphe : les personnes sont des noeuds, les amities sont des aretes. Comment trouver le chemin le plus court entre deux personnes ?
\end{histoire}

% ============================================================
\section{Qu'est-ce qu'un graphe ?}
% ============================================================

\begin{definition}[Graphe]
Ensemble de \textbf{noeuds} (sommets) relies par des \textbf{aretes} (liens).
\end{definition}

\begin{lstlisting}[caption={Representation par liste d'adjacence}]
from collections import defaultdict

graphe = defaultdict(list)
graphe["A"].append("B")
graphe["A"].append("C")
graphe["B"].append("D")
\end{lstlisting}

% ============================================================
\section{BFS : Largeur d'abord}
% ============================================================

\begin{lstlisting}[caption={BFS - Plus court chemin non pondere}]
from collections import deque

def bfs(start, goal, graphe):
    queue = deque([(start, 0)])
    visited = {start}
    
    while queue:
        node, dist = queue.popleft()
        
        if node == goal:
            return dist
        
        for voisin in graphe[node]:
            if voisin not in visited:
                visited.add(voisin)
                queue.append((voisin, dist + 1))
    
    return -1
\end{lstlisting}

% ============================================================
\section{DFS : Profondeur d'abord}
% ============================================================

\begin{lstlisting}[caption={DFS recursif}]
def dfs(node, visited, graphe):
    if node in visited:
        return
    visited.add(node)
    
    for voisin in graphe[node]:
        dfs(voisin, visited, graphe)
\end{lstlisting}

% ============================================================
\section{Dijkstra : Chemins ponderes}
% ============================================================

\begin{lstlisting}[caption={Dijkstra avec heapq}]
import heapq

def dijkstra(start, goal, graphe):
    pq = [(0, start)]
    distances = {start: 0}
    
    while pq:
        dist, node = heapq.heappop(pq)
        
        if node == goal:
            return dist
        
        if dist > distances.get(node, float('inf')):
            continue
        
        for voisin, poids in graphe[node]:
            new_dist = dist + poids
            if new_dist < distances.get(voisin, float('inf')):
                distances[voisin] = new_dist
                heapq.heappush(pq, (new_dist, voisin))
    
    return -1
\end{lstlisting}

% ============================================================
\section{Challenges d'application}
% ============================================================

\begin{exercice}
\textbf{AoC 2016 Day 13 - A Maze of Twisty Little Cubicles}

BFS dans un labyrinthe genere par une formule.

\textbf{Difficulte :} \faStar\faStar

\textbf{Concepts :} BFS, generation procedurale
\end{exercice}

\begin{exercice}
\textbf{AoC 2021 Day 15 - Chiton}

Trouve le chemin de risque minimal dans une grille.

\textbf{Difficulte :} \faStar\faStar\faStar

\textbf{Concepts :} Dijkstra, grille comme graphe
\end{exercice}

\begin{cyber}
\begin{itemize}
    \item \textbf{Scan reseau} : Cartographie de topologie
    \item \textbf{Analyse de malware} : Graphe d'appels de fonctions
    \item \textbf{Pathfinding} : Routage, navigation
\end{itemize}
\end{cyber}

\chapter{Programmation Dynamique : Optimiser par Sous-problemes}
\label{chap:dp}

\begin{histoire}
Tu veux monter un escalier. A chaque marche, tu peux faire 1 ou 2 pas. Combien de facons differentes d'arriver en haut ? Au lieu de tout recalculer, on reutilise les resultats precedents.
\end{histoire}

% ============================================================
\section{Qu'est-ce que la DP ?}
% ============================================================

\begin{definition}[Programmation Dynamique]
Technique qui resout un probleme en le decomposant en \textbf{sous-problemes} et en stockant leurs solutions pour eviter les recalculs.
\end{definition}

\begin{lstlisting}[caption={Exemple : Escalier}]
def escalier(n):
    # dp[i] = nombre de facons d'atteindre marche i
    dp = [0] * (n + 1)
    dp[0] = 1  # 1 facon de rester en bas
    dp[1] = 1  # 1 facon d'atteindre marche 1
    
    for i in range(2, n + 1):
        dp[i] = dp[i-1] + dp[i-2]
    
    return dp[n]
\end{lstlisting}

% ============================================================
\section{Les deux approches}
% ============================================================

\begin{itemize}
    \item \textbf{Top-down} : Recursion + memoization
    \item \textbf{Bottom-up} : Iteration, remplir un tableau
\end{itemize}

% ============================================================
\section{Challenges d'application}
% ============================================================

\begin{exercice}
\textbf{AoC 2020 Day 10 - Adapter Array}

Compte les arrangements possibles d'adaptateurs.

\textbf{Concepts :} DP, comptage de chemins
\end{exercice}

\begin{cyber}
\begin{itemize}
    \item \textbf{Alignement de sequences} : Comparaison ADN/malware
    \item \textbf{Optimisation} : Allocation de ressources
\end{itemize}
\end{cyber}

\chapter{Simulation : Modeliser le Monde}
\label{chap:simulation}

\begin{histoire}
Le Jeu de la Vie de Conway : des cellules naissent et meurent selon des regles simples. Pourtant, des structures complexes emergent. Bienvenue dans la simulation !
\end{histoire}

% ============================================================
\section{Automates cellulaires}
% ============================================================

\begin{lstlisting}[caption={Game of Life simplifie}]
def step(grille):
    nouvelle = set()
    
    # Compter les voisins
    for cell in grille:
        voisins = count_neighbors(cell, grille)
        if voisins in [2, 3]:
            nouvelle.add(cell)
    
    # Naissance
    for cell in candidates(grille):
        if count_neighbors(cell, grille) == 3:
            nouvelle.add(cell)
    
    return nouvelle
\end{lstlisting}

% ============================================================
\section{Detection de cycles}
% ============================================================

\begin{lstlisting}[caption={Trouver un cycle}]
def find_cycle(initial_state, step_func):
    seen = {initial_state: 0}
    state = initial_state
    step = 0
    
    while True:
        state = step_func(state)
        step += 1
        
        if state in seen:
            cycle_start = seen[state]
            cycle_length = step - cycle_start
            return cycle_start, cycle_length
        
        seen[state] = step
\end{lstlisting}

% ============================================================
\section{Challenges d'application}
% ============================================================

\begin{exercice}
\textbf{AoC 2020 Day 17 - Conway Cubes}

Game of Life en 3D et 4D.

\textbf{Concepts :} Simulation, dimensions multiples
\end{exercice}

\begin{cyber}
\begin{itemize}
    \item \textbf{Analyse de malware} : Execution symbolique
    \item \textbf{Fuzzing} : Simulation d'entrees
\end{itemize}
\end{cyber}


% === PARTIE 4 : APPLICATIONS ===
\part{Applications Cyber}

\chapter{Cryptographie : Cacher et Reveler}
\label{chap:crypto}

\begin{histoire}
Les espions ont toujours utilise des codes secrets. Aujourd'hui, c'est pareil mais avec des mathematiques. Comment cacher un message pour que seul le destinataire puisse le lire ?
\end{histoire}

% ============================================================
\section{XOR : le chiffrement reversible}
% ============================================================

\begin{definition}[XOR]
Operation binaire : 1 XOR 1 = 0, 1 XOR 0 = 1, 0 XOR 0 = 0.
Propriete magique : A XOR B XOR B = A
\end{definition}

\begin{lstlisting}[caption={Chiffrement XOR}]
def xor_encrypt(message, key):
    return bytes([m ^ k for m, k in zip(message, cycle(key))])

# Dechiffrement = meme operation !
def xor_decrypt(ciphertext, key):
    return xor_encrypt(ciphertext, key)
\end{lstlisting}

% ============================================================
\section{Hachage : l'empreinte unique}
% ============================================================

\begin{lstlisting}[caption={MD5 en Python}]
import hashlib

message = b"secret"
hash_md5 = hashlib.md5(message).hexdigest()
# '5ebe2294ecd0e0f08eab7690d2a6ee69'
\end{lstlisting}

% ============================================================
\section{Challenges d'application}
% ============================================================

\begin{exercice}
\textbf{AoC 2015 Day 4 - The Ideal Stocking Stuffer}

Trouve un nombre qui produit un hash MD5 commencant par des zeros.

\textbf{Concepts :} Hachage, bruteforce
\end{exercice}

\begin{exercice}
\textbf{AoC 2016 Day 5 - How About a Nice Game of Chess?}

Derive un mot de passe a partir de hash MD5 successifs.

\textbf{Concepts :} Hachage iteratif
\end{exercice}

\begin{cyber}
\begin{itemize}
    \item \textbf{Cracking de mots de passe} : Rainbow tables, hashcat
    \item \textbf{Integrite} : Verification de fichiers
    \item \textbf{Blockchain} : Proof of work
\end{itemize}
\end{cyber}

\chapter{Parsing Avance : Regex et Grammaires}
\label{chap:parsing_avance}

\begin{histoire}
Parfois, les donnees sont tellement complexes qu'un simple \texttt{split()} ne suffit plus. Il faut des outils plus puissants : les expressions regulieres.
\end{histoire}

% ============================================================
\section{Expressions regulieres (Regex)}
% ============================================================

\begin{lstlisting}[caption={Regex de base}]
import re

texte = "Il y a 42 pommes et 17 oranges"

# Trouver tous les nombres
re.findall(r'\d+', texte)  # ['42', '17']

# Extraire avec groupes
match = re.search(r'(\d+) pommes', texte)
match.group(1)  # '42'

# Remplacer
re.sub(r'\d+', 'X', texte)  # 'Il y a X pommes et X oranges'
\end{lstlisting}

% ============================================================
\section{Patterns courants}
% ============================================================

\begin{tabular}{ll}
\toprule
\textbf{Pattern} & \textbf{Signification} \\
\midrule
\texttt{\textbackslash d} & Chiffre \\
\texttt{\textbackslash w} & Lettre ou chiffre \\
\texttt{\textbackslash s} & Espace \\
\texttt{.} & N'importe quel caractere \\
\texttt{+} & 1 ou plus \\
\texttt{*} & 0 ou plus \\
\texttt{?} & 0 ou 1 \\
\texttt{()} & Groupe de capture \\
\bottomrule
\end{tabular}

% ============================================================
\section{Challenges d'application}
% ============================================================

\begin{exercice}
\textbf{AoC 2015 Day 5 - Doesn't He Have Intern-Elves For This?}

Valide des strings avec des regles complexes.

\textbf{Concepts :} Regex, validation
\end{exercice}

\begin{cyber}
\begin{itemize}
    \item \textbf{Log parsing} : Extraction d'IPs, timestamps
    \item \textbf{Input validation} : Securiser les formulaires
    \item \textbf{Grep/sed/awk} : Outils Unix essentiels
\end{itemize}
\end{cyber}

\chapter{Optimisation : Faire Plus Vite}
\label{chap:optimisation}

\begin{histoire}
Ton programme fonctionne, mais il met 10 minutes. Comment le faire tourner en 10 secondes ? L'optimisation, c'est l'art de faire plus avec moins.
\end{histoire}

% ============================================================
\section{Identifier les goulots}
% ============================================================

\begin{lstlisting}[caption={Profiling simple}]
import time

start = time.time()
# Code a mesurer
result = slow_function()
print(f"Temps: {time.time() - start:.2f}s")
\end{lstlisting}

% ============================================================
\section{Complexite algorithmique}
% ============================================================

\begin{tabular}{lll}
\toprule
\textbf{Complexite} & \textbf{Exemple} & \textbf{1M elements} \\
\midrule
O(1) & Acces dict & Instantane \\
O(log n) & Recherche binaire & 20 ops \\
O(n) & Parcours liste & 1M ops \\
O(n log n) & Tri & 20M ops \\
O(n\textsuperscript{2}) & Double boucle & 1T ops \\
O(2\textsuperscript{n}) & Bruteforce & Impossible \\
\bottomrule
\end{tabular}

% ============================================================
\section{Techniques courantes}
% ============================================================

\begin{itemize}
    \item Utiliser \texttt{set} au lieu de \texttt{list} pour les recherches
    \item Memoization pour eviter les recalculs
    \item Eviter les copies inutiles de listes
    \item Utiliser des generateurs au lieu de listes
\end{itemize}

% ============================================================
\section{Challenges d'application}
% ============================================================

\begin{exercice}
\textbf{AoC 2025 Day 7 - Laboratories}

Optimise une simulation exponentielle avec \texttt{defaultdict(int)}.

\textbf{Concepts :} Comptage au lieu d'enumeration
\end{exercice}

\begin{cyber}
\begin{itemize}
    \item \textbf{Timing attacks} : Mesurer les differences de temps
    \item \textbf{DoS} : Exploiter la complexite algorithmique
\end{itemize}
\end{cyber}


% === ANNEXES ===
\appendix
\part{Annexes}

\chapter{Reference Python}
\label{chap:python_ref}

% ============================================================
\section{Collections}
% ============================================================

\begin{lstlisting}[caption={Imports essentiels}]
from collections import deque, Counter, defaultdict
from itertools import permutations, combinations, product
from functools import lru_cache, reduce
import heapq
import math
import re
\end{lstlisting}

% ============================================================
\section{Comprehensions}
% ============================================================

\begin{lstlisting}[caption={Syntaxe des comprehensions}]
# Liste
[x*2 for x in range(10) if x % 2 == 0]

# Dict
{k: v for k, v in items}

# Set
{x % 3 for x in range(10)}

# Generator
(x*2 for x in range(10))
\end{lstlisting}

% ============================================================
\section{Fonctions utiles}
% ============================================================

\begin{tabular}{ll}
\toprule
\textbf{Fonction} & \textbf{Usage} \\
\midrule
\texttt{enumerate()} & Indices + valeurs \\
\texttt{zip()} & Parcourir plusieurs listes \\
\texttt{map()} & Appliquer une fonction \\
\texttt{filter()} & Filtrer des elements \\
\texttt{sorted()} & Trier (renvoie nouvelle liste) \\
\texttt{reversed()} & Inverser \\
\texttt{any()} / \texttt{all()} & Tests logiques \\
\bottomrule
\end{tabular}

\chapter{Complexite Algorithmique}
\label{chap:complexite}

% ============================================================
\section{Notation Big-O}
% ============================================================

\begin{definition}[Big-O]
La complexite decrit comment le temps d'execution evolue avec la taille de l'entree.
\end{definition}

% ============================================================
\section{Complexites courantes}
% ============================================================

\begin{tabular}{llp{6cm}}
\toprule
\textbf{Complexite} & \textbf{Nom} & \textbf{Exemples} \\
\midrule
O(1) & Constante & Acces dict/set, operations mathematiques \\
O(log n) & Logarithmique & Recherche binaire, operations sur heap \\
O(n) & Lineaire & Parcours de liste, recherche lineaire \\
O(n log n) & Linearithmique & Tri (merge sort, timsort) \\
O(n\textsuperscript{2}) & Quadratique & Double boucle imbriquee \\
O(n\textsuperscript{3}) & Cubique & Triple boucle, multiplication de matrices \\
O(2\textsuperscript{n}) & Exponentielle & Bruteforce, backtracking naif \\
O(n!) & Factorielle & Permutations completes \\
\bottomrule
\end{tabular}

% ============================================================
\section{Regles de calcul}
% ============================================================

\begin{enumerate}
    \item On garde le terme dominant : O(n\textsuperscript{2} + n) = O(n\textsuperscript{2})
    \item On ignore les constantes : O(3n) = O(n)
    \item Boucles imbriquees : on multiplie
    \item Boucles sequentielles : on additionne
\end{enumerate}

% ============================================================
\section{Complexite spatiale}
% ============================================================

\begin{definition}[Complexite spatiale]
Quantite de memoire utilisee en fonction de la taille de l'entree.
\end{definition}

\begin{itemize}
    \item Liste de n elements : O(n)
    \item Grille n x n : O(n\textsuperscript{2})
    \item Set de positions visitees : O(nombre de positions uniques)
\end{itemize}

\chapter{Index des Challenges}
\label{chap:index_challenges}

% ============================================================
\section{Par Concept Principal}
% ============================================================

\subsection{Parsing et I/O}
\begin{itemize}
    \item 2015 Day 1-5 : Parsing basique
    \item 2020 Day 4 : Parsing complexe (passeports)
    \item 2016 Day 4 : Regex
\end{itemize}

\subsection{Grilles et Navigation}
\begin{itemize}
    \item 2015 Day 3 : Set de positions
    \item 2015 Day 6 : Grille booleenne
    \item 2020 Day 11 : Game of Life
    \item 2016 Day 1 : Distance Manhattan
\end{itemize}

\subsection{BFS / Pathfinding}
\begin{itemize}
    \item 2016 Day 13 : BFS dans labyrinthe
    \item 2022 Day 12 : BFS avec contraintes
    \item 2021 Day 15 : Dijkstra
    \item 2018 Day 22 : A* avance
\end{itemize}

\subsection{Recursion et DP}
\begin{itemize}
    \item 2015 Day 7 : Recursion avec memoization
    \item 2020 Day 10 : DP comptage
    \item 2015 Day 17 : Backtracking
\end{itemize}

\subsection{Mathematiques}
\begin{itemize}
    \item 2020 Day 13 : Theoreme chinois des restes
    \item 2019 Day 12 : LCM pour cycles
    \item 2015 Day 4 : Hachage MD5
\end{itemize}

% ============================================================
\section{Par Difficulte}
% ============================================================

\subsection{Niveau Debutant (\faStar)}
2015 Days 1-5, 2020 Day 1, 2024 Day 1

\subsection{Niveau Intermediaire (\faStar\faStar)}
2016 Day 13, 2015 Day 6-10, 2020 Days 3-8

\subsection{Niveau Avance (\faStar\faStar\faStar)}
2021 Day 15, 2015 Day 7, 2020 Days 10-15

\subsection{Niveau Expert (\faStar\faStar\faStar\faStar)}
2020 Day 13 (CRT), Days 20-25 de toutes les annees

% ============================================================
\section{Challenges Emblematiques}
% ============================================================

\begin{tabular}{lp{8cm}}
\toprule
\textbf{Challenge} & \textbf{Pourquoi c'est un classique} \\
\midrule
2015 Day 7 & Introduction parfaite a la recursion avec memoization \\
2016 Day 13 & BFS pur dans un espace genere \\
2021 Day 15 & Meilleur challenge pour apprendre Dijkstra \\
2020 Day 13 & Le theoreme chinois des restes en pratique \\
2020 Day 17 & Game of Life en N dimensions \\
\bottomrule
\end{tabular}


\backmatter

\chapter*{Index des Challenges}
\addcontentsline{toc}{chapter}{Index des Challenges}

% Index par annee et par concept - a generer

\end{document}
